%%%%%%%%%%%%%%%%%%%%%%%%%%%%%%%%%%%%%%%%%
% Short Sectioned Assignment
% LaTeX Template
% Version 1.0 (5/5/12)
%
% This template has been downloaded from:
% http://www.LaTeXTemplates.com
%
% Original author:
% Frits Wenneker (http://www.howtotex.com)
%
% License:
% CC BY-NC-SA 3.0 (http://creativecommons.org/licenses/by-nc-sa/3.0/)
%
%%%%%%%%%%%%%%%%%%%%%%%%%%%%%%%%%%%%%%%%%

%----------------------------------------------------------------------------------------
%	PACKAGES AND OTHER DOCUMENT CONFIGURATIONS
%----------------------------------------------------------------------------------------

\documentclass[paper=a4, fontsize=11pt]{scrartcl} % A4 paper and 11pt font size

\usepackage[T1]{fontenc} % Use 8-bit encoding that has 256 glyphs
\usepackage{fourier} % Use the Adobe Utopia font for the document - comment this line to return to the LaTeX default
\usepackage[english]{babel} % English language/hyphenation
\usepackage{amsmath,amsfonts,amsthm} % Math packages
\usepackage{listings}
\usepackage{graphicx}
%\graphicspath{ {Desktop/PA3_numerical_analysis/} }

\usepackage{lipsum} % Used for inserting dummy 'Lorem ipsum' text into the template

\usepackage{sectsty} % Allows customizing section commands
\allsectionsfont{\centering \normalfont\scshape} % Make all sections centered, the default font and small caps

\usepackage{fancyhdr} % Custom headers and footers
\pagestyle{fancyplain} % Makes all pages in the document conform to the custom headers and footers
\fancyhead{} % No page header - if you want one, create it in the same way as the footers below
\fancyfoot[L]{} % Empty left footer
\fancyfoot[C]{} % Empty center footer
\fancyfoot[R]{\thepage} % Page numbering for right footer
\renewcommand{\headrulewidth}{0pt} % Remove header underlines
\renewcommand{\footrulewidth}{0pt} % Remove footer underlines
\setlength{\headheight}{13.6pt} % Customize the height of the header

\numberwithin{equation}{section} % Number equations within sections (i.e. 1.1, 1.2, 2.1, 2.2 instead of 1, 2, 3, 4)
\numberwithin{figure}{section} % Number figures within sections (i.e. 1.1, 1.2, 2.1, 2.2 instead of 1, 2, 3, 4)
\numberwithin{table}{section} % Number tables within sections (i.e. 1.1, 1.2, 2.1, 2.2 instead of 1, 2, 3, 4)

\setlength\parindent{0pt} % Removes all indentation from paragraphs - comment this line for an assignment with lots of text

%----------------------------------------------------------------------------------------
%	TITLE SECTION
%----------------------------------------------------------------------------------------

\newcommand{\horrule}[1]{\rule{\linewidth}{#1}} % Create horizontal rule command with 1 argument of height

\title{	
\normalfont \normalsize 
\textsc{The College of William and Mary} \\ [25pt] % Your university, school and/or department name(s)
\horrule{0.5pt} \\[0.4cm] % Thin top horizontal rule
\huge Programming Assignment \#4 \\ % The assignment title
\horrule{2pt} \\[0.5cm] % Thick bottom horizontal rule
}

\author{Alexander Powell} % Your name

\date{\normalsize December 3, 2014} % Today's date or a custom date

\begin{document}
\lstset{language=MATLAB}

\maketitle % Print the title

%----------------------------------------------------------------------------------------
%	PROBLEM 1
%----------------------------------------------------------------------------------------

\section{}





% ----------------
% PROBLEM 2
% ----------------

\section{}

The monte carlo simulation of the integral from problem $2$ was implemented using the code given below.  The built-in MATLAB functions \textit{zeros()} and \textit{sum()} were used.  





\begin{lstlisting} [frame=single]
function [ integral ] = monte_carlo( N )
a = zeros([1,N]);
for i = 1:N  
    a(i) = 2*rand - 1;
end
b = zeros([1,N]);
for i = 1:N  
    b(i) = 2*rand - 1;
end
x = a;
y = b;
temp = zeros([1,N]);
for j = 1:N
    if ((x(j).^2 + y(j).^2) < 1)
        temp(j) = 1;
    else
        temp(j) = 0;
    end
end
c = temp;
total = sum(c)
integral = 4.*(1/N).*(total);
end
\end{lstlisting}

To implement step $4$ of the Monte Carlo simulation, I calculated the integral 1000 times and took the average using the following shell commands.  

\begin{lstlisting} [frame=single]
>> t = zeros([1,1000]);
>> for i = 1:1000
>>    t(i) = monte_carlo(100000);
>> end
>> mean(t)
\end{lstlisting}


The following table presents the computed values of the integral for the given values of $N$ with their respective relative errors.  

\center{$a_j, b_j, c_j$ and $d_j$ values for $x(t)$}
%\begin{table} [H]
%\caption{Hello}
\begin{center}
  \begin{tabular}{ | c || c | c | }
    \hline
    $N$    & Integral Approx & Rel Error \\ \hline
    2      & 3.20600000      & 0.45215311 \\ \hline
    5      & 3.16320000      & 0.59569378 \\ \hline
    10     & 3.13120000      & 0.16507177 \\ \hline
    100    & 3.13736000      & 0.24401914 \\ \hline
    1000   & 3.14117200      & 0.35885167 \\
    10000  & 3.14101240      & 0.24401914 \\ \hline
    100000 & 3.14139904      & 0.24401914 \\
    \hline
  \end{tabular}
\end{center}
%\end{table}


\center{$a_j, b_j, c_j$ and $d_j$ values for $y(t)$}
%\begin{table} [H]
%\caption{other}
\begin{center}
  \begin{tabular}{ c || c | c | c | c }
    %\hline
    i & $a_i$ & $b_i$ & $c_i$ & $d_i$\\ \hline
    0 & 1   &-0.76076555 & 0          & 0.26076555 \\ \hline
    1 & 0.5 & 0.02153110 & 0.78229665 &-0.30382775 \\ \hline
    2 & 1   & 0.67464114 &-0.12918660 &-0.04545454 \\ \hline
    3 & 1.5 & 0.27990430 &-0.26555023 &-0.01435406 \\ \hline
    4 & 1.5 &-0.29425837 &-0.30861244 & 0.10287081 \\
    \hline
  \end{tabular}
\end{center}
%\end{table}




%----------------------------------------------------------------------------------------
%	PROBLEM 2
%----------------------------------------------------------------------------------------


\section{}




% -----------end document --------------------
\end{document}