%%%%%%%%%%%%%%%%%%%%%%%%%%%%%%%%%%%%%%%%%
% Short Sectioned Assignment
% LaTeX Template
% Version 1.0 (5/5/12)
%
% This template has been downloaded from:
% http://www.LaTeXTemplates.com
%
% Original author:
% Frits Wenneker (http://www.howtotex.com)
%
% License:
% CC BY-NC-SA 3.0 (http://creativecommons.org/licenses/by-nc-sa/3.0/)
%
%%%%%%%%%%%%%%%%%%%%%%%%%%%%%%%%%%%%%%%%%

%----------------------------------------------------------------------------------------
%	PACKAGES AND OTHER DOCUMENT CONFIGURATIONS
%----------------------------------------------------------------------------------------

\documentclass[paper=a4, fontsize=11pt]{scrartcl} % A4 paper and 11pt font size

\usepackage[T1]{fontenc} % Use 8-bit encoding that has 256 glyphs
\usepackage{fourier} % Use the Adobe Utopia font for the document - comment this line to return to the LaTeX default
\usepackage[english]{babel} % English language/hyphenation
\usepackage{amsmath,amsfonts,amsthm} % Math packages
\usepackage{listings}
\usepackage{graphicx}
%\graphicspath{ {Desktop/PA3_numerical_analysis/} }

\usepackage{lipsum} % Used for inserting dummy 'Lorem ipsum' text into the template

\usepackage{sectsty} % Allows customizing section commands
\allsectionsfont{\centering \normalfont\scshape} % Make all sections centered, the default font and small caps

\usepackage{fancyhdr} % Custom headers and footers
\pagestyle{fancyplain} % Makes all pages in the document conform to the custom headers and footers
\fancyhead{} % No page header - if you want one, create it in the same way as the footers below
\fancyfoot[L]{} % Empty left footer
\fancyfoot[C]{} % Empty center footer
\fancyfoot[R]{\thepage} % Page numbering for right footer
\renewcommand{\headrulewidth}{0pt} % Remove header underlines
\renewcommand{\footrulewidth}{0pt} % Remove footer underlines
\setlength{\headheight}{13.6pt} % Customize the height of the header

\numberwithin{equation}{section} % Number equations within sections (i.e. 1.1, 1.2, 2.1, 2.2 instead of 1, 2, 3, 4)
\numberwithin{figure}{section} % Number figures within sections (i.e. 1.1, 1.2, 2.1, 2.2 instead of 1, 2, 3, 4)
\numberwithin{table}{section} % Number tables within sections (i.e. 1.1, 1.2, 2.1, 2.2 instead of 1, 2, 3, 4)

\setlength\parindent{0pt} % Removes all indentation from paragraphs - comment this line for an assignment with lots of text

%----------------------------------------------------------------------------------------
%	TITLE SECTION
%----------------------------------------------------------------------------------------

\newcommand{\horrule}[1]{\rule{\linewidth}{#1}} % Create horizontal rule command with 1 argument of height

\title{	
\normalfont \normalsize 
\textsc{The College of William and Mary} \\ [25pt] % Your university, school and/or department name(s)
\horrule{0.5pt} \\[0.4cm] % Thin top horizontal rule
\huge Programming Assignment 2 \\ % The assignment title
\horrule{2pt} \\[0.5cm] % Thick bottom horizontal rule
}

\author{Alexander Powell} % Your name

\date{\normalsize April 16, 2015} % Today's date or a custom date

\begin{document}
\lstset{language=MATLAB}
\maketitle % Print the title

% -------BEGIN DOCUMENT----------%

\section{}

The following shows the modified code for F.m implemented in MATLAB

\begin{lstlisting} [frame=single]
function y = F(x)
% Computes the system of nonlinear equations from the given differential
% equation.
    y = zeros(length(x),1);
    
    a = 1;
    b = 3;
    h = 0.1;
    alpha = answ_func(a);
    beta  = answ_func(b);
    xi = zeros(((b - a)/h) + 1,1);
    N = ((b - a)/h) - 1;
    for i = 1:((b - a)/h + 1)
        xi(i,1) = a + (i - 1) * h;
    end
    
    yi = zeros(((b - a)/h) + 1, 1);
    for j = 1:(length(yi))
        yi(j,1) = answ_func(xi(j,1));
    end
    
    for k = 1:N
        if (k == 1)
            y(k) = 2 * yi(k+1) - yi(k+2) + (h^2) * func(x(1),yi(k+1),...
            (yi(k+2) - yi(k))/(2 * h)) - alpha;
        elseif (k == N)
            y(k) = (-1) * yi(k) + 2 * yi(k+1) + (h^2) * func(x(N),yi(k+1),...
            (yi(k+2) - yi(k))/(2 * h)) - beta;
        else
            y(k) = (-1) * yi(k) + 2 * yi(k+2) - yi(k+2) + (h^2) * func(x(k),...
            yi(k+1),(yi(k+2) - yi(k))/(2 * h));
        end
    end
end
\end{lstlisting}

Below is the calculation for the exact Jacobian, or exJF.m.  

\begin{lstlisting} [frame=single]
function JF = exJF( x )
% Exactly calculates the Jacobian matrix of F.  

    a = 1;
    b = 3;
    h = 0.1;
    xi = zeros(((b - a)/h) + 1,1);
    N = ((b - a)/h) - 1;
    for i = 1:((b - a)/h + 1)
        xi(i,1) = a + (i - 1) * h;
    end
    yi = zeros(((b - a)/h) + 1, 1);
    for j = 1:(length(yi))
        yi(j,1) = answ_func(xi(j,1));
    end

    JF = zeros(N,N);
    
    for i = 1:N
        for j = 1:N
            if ((i == (j - 1)) && (j >= 2) && (j <= N))
                JF(i,j) = -1 + (h/2) * fyp(x(i),yi(i+1),...
                (yi(i+2)-yi(i)/(2 * h)));
            elseif (i == j)
                JF(i,j) = 2 + (h^2) * fy(x(i),yi(i+1),...
                (yi(i+2)-yi(i)/(2 * h)));
            elseif ((i == j + 1) && (j >= 1) && (j <= (N - 1)))
                JF(i,j) = -1 - (h/2) * fyp(x(i),yi(i+1),...
                (yi(i+2)-yi(i)/(2 * h)));
            else
                JF(i,j) = 0;
            end
        end
    end    
    JF = sparse(JF);
end
\end{lstlisting}

Also, it uses the answ\textunderscore func.m function in MATLAB to calculate the exact value of the equation $$f(x) = x^2 + \frac{16}{x}$$.  Shown below is that code.  

\begin{lstlisting} [frame=single]
function out = answ_func(x)
% This function models the originial differential equation.  
    out = (x^2) + (16/x);
end
\end{lstlisting}

Furthermore, it always uses the originial differential equation, func.m calculated using the code below.  

\begin{lstlisting} [frame=single]
function out = func(x,y,yp)
    out = (1/8) * (32 + 2 * x^3 - y * yp);
end
\end{lstlisting}

The exact Jacobian is computed using the partial derivatives of $f_y$ and $f_y'$.  They are calculated in MATLAB by the following.  

\begin{lstlisting} [frame=single]
function out = fy(x,y,yp)
    out = (-1 * yp) / 8;
end
\end{lstlisting}

\begin{lstlisting} [frame=single]
function out = fyp(x,y,yp)
    out = (-1 * y) / 8;
end
\end{lstlisting}

\bigskip

When $h=0.1$ the infinity norm of the errors, defined by $|\omega_i-y(x_i)|$, is $0.002462$.  
When $h=0.01$ the infinity norm of the errors is $1.231 \times 10^{-5}$.

Our results behave like $O(h^2)$ which makes sense because if Newton's method converges, then the order of convergence is $2$.  However, we should keep in mind that the convergence of the iterative method is dependent on the given initial iterate.  


\clearpage


\section{}

Newton's method took $8$ iterations to converge for $h=0.1$ and $14$ iterations for $h=0.01$.  This makes sense because we fed in a very good approximation to the Newton iterative method.  When other iterative techniques like the Chord method and Shamanskii method were used, relatively similar results were found.  The Chord method took $9$ iterations when $h=0.1$ and $17$ when $h=0.01$.  Likewise, Shamanskii's method using $M=2$ resulted in $8$ iteration when $h=0.1$ and $12$ when $h=0.01$.  

The following table displays the times it took for the following computations to complete.  The results were calculated using MATLAB's built-in tic and toc methods.  

\begin{table}[ht]
\begin{center}
\caption {Iteration Times for Algorithms}
\begin{tabular} {| c | c | c | c |}
\hline
$h$ & Newton & Chord & Shamanskii \\
\hline
$0.1$  & $0.00348$ & $0.002876$ & $0.047709$ \\
$0.01$ & $0.01203$ & $0.038763$ & $0.134587$ \\
\hline
\end{tabular}
\end{center}
\end{table}

Based on these rates of convergence, all methods do a pretty good job given the initial iterate.  If any of the three was slower than the others it would probably by Shamanskii's algorithm, as it took a few more iterations to converge.  Also, it should be noted that all of these results were found using the tolerance as $10^{-8}$.  







% --------END DOCUMENT-----------%

\end{document}